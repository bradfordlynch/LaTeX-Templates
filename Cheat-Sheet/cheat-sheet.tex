% document properties
\documentclass[10pt,landscape]{article}
\title{Metrics Cheat Sheet}
\author{Travis Cao}

% The following LaTeX setup is borrowed from Rebekah Dix (GitHub: rebekahanne)

%%%%%%%%%%%%%%%%%%%%%%%%%%%%%%%%%%%%%%%%%%%%%%%%%%%%%%%%%%%%%%%%

% Packages

% AMS 
\usepackage{amsmath, amssymb, amsthm, amsbsy}
% Geometry
\usepackage[landscape]{geometry}
% Colors
\usepackage[usenames,dvipsnames]{xcolor}
% Figures
\usepackage{float}
\usepackage{graphicx}
% Multi column lists
\usepackage{multicol}
% Pretty tables
\usepackage{booktabs}
% Subfigures
\usepackage{caption}
\usepackage{subcaption}
% Caligraphic
\usepackage{mathrsfs}
\usepackage{bbm}
% Bold
\usepackage{bm}
% algos
\usepackage[linesnumbered, lined, ruled]{algorithm2e}
% Spacing 
\usepackage{setspace}
% Backslash box
\usepackage{diagbox}
% Multirow in table
\usepackage{multirow}
% Refs
\usepackage[colorlinks=true, citecolor=Blue, linkcolor=blue]{hyperref}
\newcommand\myshade{85}
\colorlet{mylinkcolor}{violet}
\colorlet{mycitecolor}{PineGreen}
\colorlet{myurlcolor}{Aquamarine}
\hypersetup{
	colorlinks = true,
  	linkcolor  = mylinkcolor!\myshade!black,
 	citecolor  = mycitecolor!\myshade!black,
  	urlcolor   = myurlcolor!\myshade!black
}
% Bibliography
\usepackage{filecontents}
\usepackage{natbib}
% Indent
\usepackage{indentfirst}
%\setlength\parindent{0pt}
%\setlength{\parskip}{7pt} % vertical space between paragraphs
% Pretty lists
\usepackage{enumitem}
\setlist[enumerate]{itemsep=-2pt,topsep=2pt}
\setlist[itemize]{itemsep=0pt,topsep=1pt}
\setlist[enumerate,1]{label=(\roman*)}
% Code
\usepackage{listings}
% Appendix
\usepackage[toc,page]{appendix}
% Math
\usepackage{mathtools}
\usepackage{xparse}
% Equation numbering
\numberwithin{equation}{section}
% Use more than one optional parameter in a new commands
\usepackage{xargs}                      
% Todo
\usepackage[colorinlistoftodos,prependcaption,textsize=normalsize]{todonotes}
\newcommandx{\unsure}[2][1=]{\todo[linecolor=red,backgroundcolor=red!25,bordercolor=red,#1]{#2}}
\newcommandx{\change}[2][1=]{\todo[linecolor=blue,backgroundcolor=blue!25,bordercolor=blue,#1]{#2}}
\newcommandx{\info}[2][1=]{\todo[linecolor=OliveGreen,backgroundcolor=OliveGreen!25,bordercolor=OliveGreen,#1]{#2}}
\newcommandx{\improvement}[2][1=]{\todo[linecolor=Plum,backgroundcolor=Plum!25,bordercolor=Plum,#1]{#2}}
\newcommandx{\thiswillnotshow}[2][1=]{\todo[disable,#1]{#2}}
% Framed theorems
\usepackage[nobreak=true,align=center,userdefinedwidth=30em,linewidth=1pt]{mdframed}
% Change left or right spacing
\usepackage{changepage}
% Footnote symbol
\usepackage{footmisc}
\renewcommand{\thefootnote}{\fnsymbol{footnote}}

%%%%%%%%%%%%%%%%%%%%%%%%%%%%%%%%%%%%%%%%%%%%%%%%%%%%%%%%%%%%%%%%

% Document Settings

% Figure path
\graphicspath{{./figures/}}
% Matrix columns
\setcounter{MaxMatrixCols}{10}
% So pages will break inside long equation environments
\allowdisplaybreaks
% Font
\usepackage{mathpazo} 
\linespread{1.05}  
%\usepackage{courier}
% Geometry
\geometry{left=.5in,right=.5in,top=.5in,bottom=.5in}
% Counters
\setcounter{tocdepth}{2}
\setcounter{secnumdepth}{3}

%%%%%%%%%%%%%%%%%%%%%%%%%%%%%%%%%%%%%%%%%%%%%%%%%%%%%%%%%%%%%%%%

% Cheatsheet Specific Styles

% Turn off header and footer
\pagestyle{empty}
% Redefine section commands to use less space
\makeatletter
\renewcommand{\section}{\@startsection{section}{1}{0mm}
                        {-1ex plus -.5ex minus -.2ex}
                        {0.5ex plus .2ex}
                        {\vspace{.8cm}\normalfont\large\bfseries\underline}}
\renewcommand{\subsection}{\@startsection{subsection}{2}{0mm}
                           {-1explus -.5ex minus -.2ex}
                           {0.5ex plus .2ex}{\normalfont\normalsize\bfseries}}
\renewcommand{\subsubsection}{\@startsection{subsubsection}{3}{0mm}
                              {-1ex plus -.5ex minus -.2ex}
                              {1ex plus .2ex}
                              {\normalfont\small\bfseries}}
\makeatother
% Don't print section numbers
\setcounter{secnumdepth}{0}

\setlength{\parindent}{0pt}
\setlength{\parskip}{0pt plus 0.5ex}



%%%%%%%%%%%%%%%%%%%%%%%%%%%%%%%%%%%%%%%%%%%%%%%%%%%%%%%%%%%%%%%%

% Colors
\definecolor{Tm}{rgb}{0,0,0.80}
\newcommand{\navy}[1]{\textcolor{MidnightBlue}{\bf #1}}

%%%%%%%%%%%%%%%%%%%%%%%%%%%%%%%%%%%%%%%%%%%%%%%%%%%%%%%%%%%%%%%%

% Boxed-up theorems
\theoremstyle{definition}
\mdfdefinestyle{tightTop}{
    linecolor=black,
    linewidth=.7pt,
    innertopmargin=0pt,
    nobreak=false
}
\newmdtheoremenv[style=tightTop]{definition}{\color{MidnightBlue}{\textbf{Definition}}}[subsection]
\newmdtheoremenv[style=tightTop]{theorem}{\color{Maroon}{\textbf{Theorem}}}[subsection]
\newmdtheoremenv[style=tightTop]{lemma}[theorem]{\color{ForestGreen}{\textbf{Lemma}}}
\newmdtheoremenv[style=tightTop]{proposition}[theorem]{\color{ForestGreen}{\textbf{Proposition}}}
\newmdtheoremenv[style=tightTop]{corollary}[theorem]{\color{ForestGreen}{\textbf{Corollary}}}
\newmdtheoremenv[style=tightTop]{modelenvironment}[theorem]{\color{MidnightBlue}{\textbf{Model Environment}}}

% Generic box
\newenvironment{boxedup}{\begin{mdframed}[style=tightTop, innertopmargin=8pt]}{\end{mdframed}}

% Non-boxed-up theorems
\newtheorem{claim}{\color{ForestGreen}{\textbf{Claim}}}
\newtheorem{axiom}[theorem]{\color{ForestGreen}{\textbf{Axiom}}}
\newtheorem{conjecture}[theorem]{Conjecture}
\newtheorem{case}[theorem]{Case}
\newtheorem{conclusion}[theorem]{Conclusion}
\newtheorem{criterion}[theorem]{Criterion}
\newtheorem{notation}[theorem]{Notation}
\newtheorem{problem}[theorem]{Problem}
\newtheorem{example}{\color{WildStrawberry}Example}[subsection]
\newtheorem{assumption}{Assumption}[subsection]
\newtheorem{condition}[assumption]{Condition}
\newtheorem*{solution}{\color{Goldenrod}Solution}
\newtheorem{exercise}{\color{YellowOrange}Exercise}[subsection]

%%%%%%%%%%%%%%%%%%%%%%%%%%%%%%%%%%%%%%%%%%%%%%%%%%%%%%%%%%%%%%%%

% Math macros

% Math ``brackets''
\newcommand\parens[1]{\left( #1 \right)}
\newcommand\squares[1]{\left[ #1 \right]}
\newcommand\braces[1]{\left\{ #1 \right\}}
\newcommand\angles[1]{\left\langle #1 \right\rangle}
\newcommand\ceil[1]{\left\lceil #1 \right\rceil}
\newcommand\floor[1]{\left\lfloor #1 \right\rfloor}
\newcommand\abs[1]{\left| #1 \right|}
\newcommand\dabs[1]{\left\| #1 \right\|}
\newcommand\vect[1]{\mathbf{#1}}
\newcommand\closure[1]{\overline{#1}}
\newcommand\pset[1]{\mathcal{P}\left(#1\right)}
\newcommand\inv[1]{#1^{-1}}
\newcommand\norm[1]{\lVert#1\rVert}

% inner product
\providecommand{\inner}[1]{\left\langle{#1}\right\rangle}
% stochastic dominance
\newcommand{\lesd}{\preceq_{\textrm{SD}}}
% Set builder (use \Set ultimately and separate by ;)
\DeclarePairedDelimiterX{\set}[1]{\{}{\}}{\setargs{#1}}
\NewDocumentCommand{\setargs}{>{\SplitArgument{1}{;}}m}
{\setargsaux#1}
\NewDocumentCommand{\setargsaux}{mm}
{\IfNoValueTF{#2}{#1} {#1\nonscript\:\delimsize\vert\allowbreak\nonscript\:\mathopen{}#2}}%
\def\Set{\set*}%

% center math cell
\newcommand*\centermathcell[1]{\omit\hfil$\displaystyle#1$\hfil\ignorespaces}

% hat
\usepackage{scalerel,stackengine}
\stackMath
\newcommand\widehatt[1]{%
\savestack{\tmpbox}{\stretchto{%
  \scaleto{%
    \scalerel*[\widthof{\ensuremath{#1}}]{\kern-.6pt\bigwedge\kern-.6pt}%
    {\rule[-\textheight/2]{1ex}{\textheight}}%WIDTH-LIMITED BIG WEDGE
  }{\textheight}% 
}{0.5ex}}%
\stackon[1pt]{#1}{\tmpbox}%
}

% Proof logic
\newcommand{\then}{$\Rightarrow$ }
\newcommand{\suf}{$\Leftarrow$ }
\newcommand{\nec}{$\Rightarrow$ }

% Shortcut math
\newcommand{\ls}{\leqslant}
\newcommand{\gs}{\geqslant}
\def\ss{\subset}
\def\sse{\subseteq}
\def\nss{\not \ss}
\def\sps{\supset}
\def\pss{\subsetneq}
\def\prece{\preccurlyeq}
\def\condgap{\hspace{1cm}}
\def\eprec{\preceq}
% argmax and min
\newcommand{\argmax}{\operatornamewithlimits{argmax}}
\newcommand{\argmin}{\operatornamewithlimits{argmin}}
\newcommand{\es}{\emptyset}
% Implication and reverse implication
\def\imp{\Rightarrow}
\def\pmi{\Leftarrow}
% Integers up to number
\newcommand\intsfin[1]{\braces{1, \ldots, #1}}
% Logic
\def\bic{\Leftrightarrow}
% Bold and italic
\newcommand\boldit[1]{\textbf{\textit{#1}}}
% Misc math
\newcommand{\st}{\ensuremath{\ \mathrm{s.t.}\ }}
\newcommand{\setntn}[2]{ \{ #1 : #2 \} }
\newcommand{\cf}[1]{ \lstinline|#1| }
\newcommand{\fore}{\therefore \quad}
\newcommand{\tod}{\stackrel { d } {\to} }
\newcommand{\tow}{\stackrel { w } {\to} }
\newcommand{\toprob}{\stackrel { p } {\to} }
\newcommand{\toms}{\stackrel { ms } {\to} }
\newcommand{\eqdist}{\stackrel{d} {=} }
\newcommand{\iidsim}{\stackrel{\textrm{ {\sc iid }}} {\sim} }
\newcommand{\1}{\mathbbm 1}
\newcommand{\dee}{\,{\rm d}}
\newcommand{\given}{\, | \,}
\newcommand{\la}{\langle}
\newcommand{\ra}{\rangle}

% Shortcut greek
\def\a{\alpha}
\def\b{\beta}
\def\g{\gamma}
\def\D{\Delta}
\def\d{\delta}
\def\z{\zeta}
\def\k{\kappa}
\def\l{\lambda}
\def\n{\nu}
\def\r{\rho}
\def\s{\sigma}
\def\t{\tau}
\def\x{\xi}
\def\w{\omega}
\def\W{\Omega}
% Nice greek
\newcommand{\p}{\varphi}
\newcommand{\e}{\varepsilon}

% Shorcut vectors
\def\vx{\vect{x}}
\def\vy{\vect{y}}
\def\va{\vect{a}}
\def\vb{\vect{b}}

\newcommand{\CC}{\mathbb C}
\newcommand{\FF}{\mathbb F}
\newcommand{\RR}{\mathbb R}
\newcommand{\NN}{\mathbb N}
\newcommand{\PP}{\mathbbm P}
\newcommand{\EE}{\mathbbm E}
\newcommand{\TT}{\mathbbm T}
\newcommand{\VV}{\mathbbm V}
\newcommand{\QQ}{\mathbbm Q}
\newcommand{\WW}{\mathbbm W}
\newcommand{\ZZ}{\mathbbm Z}
\newcommand{\KK}{\mathbbm K}
\renewcommand{\SS}{\mathbbm S}
\newcommand{\plim}{\text{plim}}

% Expectation/Probability
\newcommand{\ee}[1]{\mathbbm{E}[{#1}]}
\newcommand{\pp}[1]{\mathbbm{P}({#1})}

\newcommand{\GG}{\mathsf G}
\newcommand{\XX}{\mathsf X}
\renewcommand{\AA}{\mathsf A}
\newcommand{\YY}{\mathsf Y}
\newcommand{\ZZZ}{\mathsf Z}

\newcommand{\cC}{\mathscr C}
\newcommand{\iI}{\mathscr I}
\newcommand{\eE}{\mathscr E}
\newcommand{\fF}{\mathscr F}
\newcommand{\rR}{\mathscr R}
\newcommand{\sS}{\mathscr S}
\newcommand{\lL}{\mathscr L}
\newcommand{\cG}{\mathscr G}

\newcommand{\aA}{\mathcal A}
\newcommand{\pP}{\mathcal P}
\newcommand{\vV}{\mathcal V}
\newcommand{\dD}{\mathcal D}
\newcommand{\mM}{\mathcal M}
\newcommand{\oO}{\mathcal O}
\newcommand{\gG}{\mathcal G}
\newcommand{\hH}{\mathcal H}
\newcommand{\tT}{\mathcal T}
\newcommand{\bB}{\mathcal B}

% Common collections
\def\cA{\col{A}}
\def\cB{\col{B}}
\def\cC{\col{C}}
\def\cT{\col{T}}
\def\cU{\col{U}}

% Common closures
\def\clA{\closure{A}}
\def\clB{\closure{B}}
\def\clK{\closure{K}}

% operators
\DeclareMathOperator{\cl}{cl}
\DeclareMathOperator{\graph}{graph}
\DeclareMathOperator{\interior}{int}
\DeclareMathOperator{\Prob}{Prob}
\DeclareMathOperator{\determinant}{det}
\DeclareMathOperator{\trace}{trace}
\DeclareMathOperator{\sgn}{sgn}
\DeclareMathOperator{\Span}{span}
\DeclareMathOperator{\diag}{diag}
\DeclareMathOperator{\proj}{proj}
\DeclareMathOperator{\rank}{rank}
\DeclareMathOperator{\cov}{Cov}
\DeclareMathOperator{\corr}{Corr}
\DeclareMathOperator{\var}{Var}
\DeclareMathOperator{\mse}{mse}
\DeclareMathOperator{\se}{se}
\DeclareMathOperator{\row}{row}
\DeclareMathOperator{\col}{col}
\DeclareMathOperator{\range}{rng}
\DeclareMathOperator{\kernel}{ker}
\DeclareMathOperator{\dimension}{dim}
\DeclareMathOperator{\bias}{bias}
\DeclareMathOperator{\dom}{dom}
\DeclareMathOperator{\ran}{ran}
\DeclareMathOperator{\Int}{Int}
\DeclareMathOperator{\Cl}{Cl}

%%%%%%%%%%%%%%%%%%%%%%%%%%%%%%%%%%%%%%%%%%%%%%%%%%%%%%%%%%%%%%%%

\begin{document}

\raggedright
\footnotesize
\begin{multicols}{3}

% multicol parameters
% These lengths are set only within the two main columns
%\setlength{\columnseprule}{0.25pt}
\setlength{\premulticols}{1pt}
\setlength{\postmulticols}{1pt}
\setlength{\multicolsep}{1pt}
\setlength{\columnsep}{2pt}

%%%%%%%%%%%%%%%%%%%%%%%%%%%%%%%%%%%%%%%%%%%%%%%%%%%%%%%%%%%%%%%%

\begin{center}
    \large{710a Metrics Notes} \\
    \vspace{.1cm}
    \normalsize{Travis Cao}
\end{center}

\vspace{-.8cm}

\section{Stats}

\subsection{Geometric series sum}
$\sum_{j = 0}^n r^j = \frac{1 - r^{n+1}}{1 - r}$

\subsection{Conditional expectation}
$E[Y] = \sum_l E[Y|Z = l] Pr(Z = l)$

\subsection{Block inversion formula}
$M$ is invertible iff $A-BD^{-1}C$ is invertible, and $M^{-1} = $
\begin{align*}
  \begin{bmatrix}
    (A-BD^{-1}C)^{-1} & -(A-BD^{-1}C)^{-1}BD^{-1} \\ 
    -D^{-1}C(A-BD^{-1}C)^{-1} & D^{-1}+D^{-1}C(A-BD^{-1}C)^{-1}BD^{-1}
  \end{bmatrix}
\end{align*}

\subsection{Sherman-Morrison formula}
$A+uv'$ is invertible iff $1+v'A^{-1}u \neq 0$, and
\begin{align*}
  (A+uv')^{-1} = A^{-1} - \frac{A^{-1}uv'A^{-1}}{1+v'A^{-1}u}
\end{align*}

\subsection{Partial and average partial effect}
\begin{itemize}
  \item Partial effect (at $X = x$): $\frac{\partial}{\partial x} E[Y|X = x]$
  \item Average partial effect: $E[\frac{\partial}{\partial X} E[Y|X]]$
\end{itemize}

\subsection{Inequalities}
\begin{itemize}
  \item Chebyshev's: $Pr (\abs{X - E[X]} \geq \epsilon) \leq \frac{Var(X)}{\epsilon^2}$ for all $\epsilon > 0$
  \item Markov: $Pr(X \geq \epsilon) \leq \frac{E[X]}{\epsilon}$ for all $\epsilon > 0$
  \item Jensen's: For convex function $g(\cdot)$, $g(E[X]) \leq E[g(X)]$
  \item Cauchy Schwarz: $E[XY]^2 \leq E[X^2] E[Y^2]$
\end{itemize}


\subsection{Martingale CLT}
If $\{ Z_t \}_{t=1}^T$ satisfy
\begin{itemize}
  \vspace{3pt}
  \item $\{ Z_t \}_{t=1}^T$ is strictly stationary
  \vspace{3pt}
  \item $E[Z_1^2] < \infty$
  \vspace{3pt}
  \item $E[Z_t | Z_{t-1}, Z_{t-2}, \ldots, Z_1] = 0$
  \vspace{2pt}
  \item $\frac{1}{T} \sum_{t=1}^T Z_t^2 \xrightarrow{p} E[Z_1^2]$
\end{itemize}
then $\frac{1}{\sqrt{T}} \sum_{t=1}^T Z_t \xrightarrow{d} N(0, E[Z_1^2])$ as $T \rightarrow \infty$

% \vfill\null
% \columnbreak

\section{IV}

Model: $Y_i = X_i'\beta + U_i$

Assumptions: 
\begin{itemize}
  \item Validity of instrument
  \begin{itemize}
    \item Exogeneity: $E[U | Z] = 0$
    \item Relevance: $E[Z X']$ and $\sum_{i=1}^n Z_i X_i'$ are invertible
  \end{itemize}
  \item $E[|Y|^2 + ||X||^2 + ||Z||^2] < \infty$
  \item $\{ (Y_i, X_i', Z_i') \}$ are i.i.d.
\end{itemize}

IV estimator: $\hat{\beta}_{IV} = (\sum_{i=1}^n Z_i X_i')^{-1} (\sum_{i=1}^n Z_i Y_i')$

\subsection{Small sample property}
Model: $Y_i = \beta_0 + X_i \beta_1 + U_i$

\hspace{24pt} $X_i = \pi_0 + Z_i \pi_1 + V_i$

\vspace{3pt}
\hspace{24pt} $
  \begin{pmatrix}
    U_i \\ 
    V_i
  \end{pmatrix} \bigg| \enspace Z_i \sim N \left( 
    \begin{pmatrix}
      0 \\ 
      0
    \end{pmatrix}, 
    \begin{pmatrix}
      \sigma_u^2 & \sigma_{uv} \\ 
      \sigma_{uv} & \sigma_v^2
    \end{pmatrix} \right)
$
\vspace{3pt}

\then $E[\hat{\beta}_{IV} | \mathbbmss{X}, \mathbbmss{Z}] = \beta + \frac{\sigma_{uv}}{\sigma_v^2} \frac{\nu}{\frac{\hat{\pi}_1}{se(\hat{\pi}_1)} + \nu}$, where $\nu \sim N(0, 1)$

\subsection{Large sample property}

Large sample approximation is reasonable depends on both 
\begin{itemize}
  \item sample size $n$ (the bigger, the better), and
  \item the covariance between the instrument and endogenous regressor $\pi_1$ (the bigger, the better)
\end{itemize}

Which means there's a need of testing whether the instrument is relevant (whether $\pi_1 \neq 0$)

\begin{adjustwidth}{.3cm}{0cm}
  Model: $X_1 = Z_1 \pi_1 + X_2' \pi_2 + V$. Let $\hat{\pi}_1$ be OLS estimate of $\pi_1$. 

  Then the F-statistic is $F = \frac{(\hat{\pi}_1 - 0)^2}{se({\hat{\pi}_1})^2} = \frac{\hat{\pi}_1^2}{se({\hat{\pi}_1})^2}$

  Empirically, if $F \geq 10$, then some rough insurance with nominal $95\%$ confidence intervals have actual coverage of at least $80\%$.
\end{adjustwidth}

\subsection{Weak instrument}

Model: $Y = X_1 \beta_1 + U$. Weak instrument is $Z_1$. 

Assumptions on $Z_1$: 
\begin{itemize}
  \item Exogeneity: $E[U|Z_1] = 0$
  \item Allow for relevance
  \item $E[Z_1^2]$ and $\frac{1}{n} \sum_i Z_{1i}^2$ are both non-zero (i.e. invertible)
\end{itemize}

Under weak instrument, want to test
\begin{align*}
  H_0 : \beta_1 = c \quad \quad H_1 : \beta_1 \neq c
\end{align*}

Identification under $H_0$: $E[Z_1(Y - X_1 c)] = 0$

\then $T = \frac{1}{n} \sum_i Z_{1i} (Y_i - X_{1i}c)$. Reject $H_0$ when $\abs{T}$ is large

Anderson-Rubin Test: 

\begin{adjustwidth}{.3cm}{0cm}
$AR \coloneqq \frac{\sqrt{n}T}{S} \xrightarrow{d} N(0, 1)$

where $S^2 = \frac{1}{n} Z_{1i}^2 (Y_i - X_{1i}c)^2 \xrightarrow{p} E[Z_1^2 U^2]$

With a size of $5\%$, reject $H_0$ when $\abs{AR} > 1.96$

Confidence set: $\{ c \in \RR : \abs{AR(c)} \leq 1.96 \}$
\end{adjustwidth}

\subsection{Optimal instrument}

Say optimal instrument is $h^*(Z)$. Identification yields from $E[h(Z) U] = 0$. So 
$
  \hat{\beta}^h_{IV} = (\frac{1}{n} \sum_{i} h(Z_i)X_i' )^{-1} \frac{1}{n}\sum_{i} h(Z_i) Y_i
$

Under homoskedasticity, asymptotic variance (AVAR) is 
\begin{align*}
  \Omega^h &= \frac{E[h(Z)^2]}{E[h(Z)X]^2} \sigma_u^2 = \frac{E[h(Z)^2]}{E[h(Z) E[X|Z]]^2} \sigma_u^2 \\
  &\geq \frac{E[h(Z)^2]}{E[h(Z)]^2 E[E[X|Z]^2]} \sigma_u^2 = \frac{\sigma_u^2}{E[E[X|Z]^2]}
\end{align*}
To achieve the lower bound of AVAR, 
$
  h^*(Z) = E[X|Z]
$

\subsection{Random coefficient model}

Model: $Y = X'\beta_0 \cdot U = X'\beta_0 + X'\beta_0 \cdot (U - 1)$

Assumptions: 
\begin{itemize}
  \item $E[U|Z] = 1$, and $E[ZX']$ invertible
  \item $Z$ independent of $Y$
  \item Let $X = (1, X)'$, $Z = (1, Z)'$, $Z \in \{ 0, 1 \}$, $X \in \{ 0, 1 \}$
\end{itemize}
\begin{align*}
  \hat{\beta}_{IV} \xrightarrow{p} \frac{Cov(Z, Y)}{Cov(Z, X)} = \frac{E[Y|Z = 1] - E[Y|Z = 0]}{E[X|Z = 1] - E[X|Z = 0]}
\end{align*}
For binary $X$, its response to $Z$ given unobservable $U$ is $X_U(Z)$
\begin{center}
  \begin{tabular}{c| c c}
    \toprule
    & $X_U(0) = 0$ & $X_U(0) = 1$ \\
    \midrule
    $X_U(1) = 0$ & Never taking & Defying \\
    $X_U(1) = 1$ & Complying & Always taking \\
    \bottomrule
  \end{tabular}
\end{center}
Assume no defyers, and $Pr(\text{Complying}) > 0$, then
\begin{align*}
  &E[X|Z = 1] - E[X|Z = 0] = E[X_U(1) - X_U(0)] \\ 
  &= Pr(X_U(1) - X_U(0) = 1) - Pr(X_U(1) - X_U(0) = -1) \\
  &= Pr(\text{Complying})\\ 
  &E[Y|Z = 1] - E[Y|Z = 0] = E[Y_U(1)X_U(1) + Y_U(0)(1-X_U(1))] \\
  &\quad - E[Y_U(1)X_U(0) + Y_U(0)(1-X_U(0))] \tag{since $Y = Y_U(1)X + Y_U(0)(1-X)$}\\
  &= E[(Y_U(1) - Y_U(0))(X_U(1) - X_U(0))] \\ 
  &= E[(Y_U(1) - Y_U(0))|X_U(1) - X_U(0) = 1] \times Pr(\text{Complying})
\end{align*}
Thus, $\hat{\beta}_{IV} \xrightarrow{p} E[\underbrace{(Y_U(1) - Y_U(0))}_{\text{mean response of $Y$ to $X$}}|\underbrace{X_U(1) - X_U(0) = 1}_{\text{for complyers}}]$

% \vfill\null
% \columnbreak

\section{Time Series}

\subsection{TS Models}
\begin{itemize}
  \item Static: $Y_t = \alpha_0 + X_t' \delta_0 + U_t$
  \item FDL($s$): $Y_t = \alpha_0 + X_t' \delta_0 + \ldots + X_{t-s}' \delta_s + U_t$ 
  \item AR($p$): $Y_t = \alpha_0 + Y_{t-1}\rho_1 + \ldots + Y_{t-p}\rho_p + U_t$
  \item MA($q$): $Y_t = \e_t + \theta_1 \e_{t-1} + \ldots + \theta_q \e_{t-q}$
  \item Trend: 
    \begin{itemize}
      \item Linear time trend: $Y_t = \beta_0 + \beta_1 t + U_t$
      \item Exponential trend: $\log(Y_t) = \beta_0 + \beta_1 t + U_t$
    \end{itemize}
  \item Seasonality: $Y_t = \alpha_0 + \alpha_1 1_{\{t/12 \text{ is an integer}\}} + U_t$
\end{itemize}

\subsection{Stationarity}
\begin{itemize}
  \item Strict: $(Y_{t_1}, Y_{t_2}, \ldots, Y_{t_k}) \sim (Y_{t_1+l}, Y_{t_2+l}, \ldots, Y_{t_k+l})$
  
  \vspace{-1.5pt}
  (Joint distribution is $t$ independent)

  \item Weak (Covariance): For all $t$, $E[Y_t]$ and $\gamma(k) = Cov(Y_t, Y_{t+k})$ ($\leftarrow$ autocovariance function) are both independent of $t$
\end{itemize}

\end{multicols}

\end{document}